\section*{Introduction}

Monitoring plays an important role in todays offerings of cloud computing because it enables cloud providers and consumers to check if the quality of service they have agreed upon is satisfied, additionally monitoring can be used to verify that a certain level of security is enabled, by checking different host parameters.

Resources in the cloud can be provisioned and released with ease, in a short period of time, and monitoring systems should be able to scale just as fast in order to adapt to the new monitoring requirements. In order to adapt, the monitoring systems should relay on a resource allocation mechanism that can take into account the architectures of the monitoring systems and provide strategies for resource allocation. The architecture of the monitoring systems is very important because it dictates the means of obtaining scalability. Some systems have architectures that allow scaling because they are build from decoupled components that can be scaled independently, very similar to the micro-services architecture \footnote{http://microservices.io/patterns/microservices.html}, while other systems are build on a monolithic architecture making scaling more challenging and in some cases even impossible.

In this report we will focus on a monitoring system that implements a pipe and filter architecture allowing independent scaling, via replication, for all components, and thus providing fault tolerance and parallel execution of jobs.
