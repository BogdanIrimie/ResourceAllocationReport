\section*{Introduction}

Monitoring plays an important role in today's offerings of cloud computing because it enables cloud providers and consumers to check if the quality of service they have agreed upon is satisfied. Additionally monitoring can be used to verify that a certain level of security is enabled, by checking different host parameters.

Resources in the cloud can be provisioned and released with ease, in a short period of time, and monitoring systems should be able to scale just as fast in order to adapt to the new monitoring requirements. In order to adapt, the monitoring systems should relay on a resource allocation mechanism that can take into account the architectures of the monitoring systems and provide strategies for resource allocation. The architecture of the monitoring systems is very important because it dictates the means of obtaining scalability. Some systems have architectures that allow scaling because they are build from decoupled components that can be scaled independently, very similar to the micro-services architecture \footnote{http://microservices.io/patterns/microservices.html}, while other systems are build on a monolithic architecture making scaling more challenging and in some cases even impossible. For systems that are built from multiple components , like the ones that fallow a pipe and filter architecture, resource allocation can be a difficult task because components have little or no knowledge about the rest of the system.


In this paper we will focus on building different architectures for a pipe and filter system in order to allow the system to automatically scale depending on the load and the quality of service we want to enforce. System scaling is obtained via replication for all components. Replication does not only permit us to ensure scalability, but fault tolerance as well.

The rest of the paper is structures as fallows. Section \ref{sec:relatedWork} presents similar efforts to provide resource allocation while section \ref{sec:architecture} describes the pipe and filter architecture and the system built using this architecture. Resource allocation strategies and the architectural changes required are described in Section \ref{sec:resourceAllocation}. Gathering of performance metrics is presented in Section \ref{sec:performance} and we conclude with conclusions and further work that can be done.
