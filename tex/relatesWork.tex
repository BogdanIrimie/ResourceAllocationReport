\section{Related work}
\label{sec:relatesWork}

In \cite{Hussain2013} the authors survey the resource allocation efforts for distributed systems. They categorisers the systems in three types: cluster, grid and cloud. Characteristics are analysed for each category and systems are included in categories based on their characteristics. The authors of \cite{Dubois} present a proactive resource allocation mechanism based on extended queueing network models that is able to decide what resources to rent in the cloud in order to run a application that respects a set of quality of service. The authors used AWS spot instances and tried to find the appropriate bid in order to keep the VM alive and achieve low costs of renting.

LINE \cite{Perez2013} tool is a solver for queuing network models and can be used in order to evaluate performance parameters like response time for complex applications. Using this tool and models of the system, we can predict the performance of the system and know what resources should be allocated in order to achieve the desired quality of service.

A cloud controller, FQL4KE, that has self-adaptive and self-learning capabilities is presented in \cite{Jamshidi2015}. The cloud controller tries to adjust scaling policies at runtime by combining fuzzy control and Fuzzy Q-Learning. The approach mitigates the drawbacks of reactive resource allocation and proactive allocation by enhancing the fuzzy controller with machine learning capabilities.
