\section*{Conclusions}
The paper presents extensions for a system that is built on a pipe and filter architecture. Four possible architectures Fig. \ref{fig:randomDistributionsOfTasks} \ref{fig:sizeBaseDistributionOftasks} \ref{fig:multipleQueueDistributionOfTasks} \ref{fig:mapDistributionOfTasks} were suggested, each one with its advantages and disadvantages. Several ways of monitoring performance were tested and a monitoring architecture was built around "atop" software tool. 

The work from this paper bring to monitoring system one step closer to the main objective of dynamic allocation of components on virtual machines. Constant change in resource utilization of each component replica must be taken into account in order to predict the VM where to allocate a component replica and continuous performance monitoring must be implemented in order to record the change in resource utilization of each component replica. The next step is to build a component that is able to use the performance metrics gathered and make allocation plans taking into account the constraints of the architecture of the system.